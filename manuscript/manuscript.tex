\documentclass{article}\usepackage[]{graphicx}\usepackage[]{color}
% maxwidth is the original width if it is less than linewidth
% otherwise use linewidth (to make sure the graphics do not exceed the margin)
\makeatletter
\def\maxwidth{ %
  \ifdim\Gin@nat@width>\linewidth
    \linewidth
  \else
    \Gin@nat@width
  \fi
}
\makeatother

\definecolor{fgcolor}{rgb}{0.345, 0.345, 0.345}
\newcommand{\hlnum}[1]{\textcolor[rgb]{0.686,0.059,0.569}{#1}}%
\newcommand{\hlstr}[1]{\textcolor[rgb]{0.192,0.494,0.8}{#1}}%
\newcommand{\hlcom}[1]{\textcolor[rgb]{0.678,0.584,0.686}{\textit{#1}}}%
\newcommand{\hlopt}[1]{\textcolor[rgb]{0,0,0}{#1}}%
\newcommand{\hlstd}[1]{\textcolor[rgb]{0.345,0.345,0.345}{#1}}%
\newcommand{\hlkwa}[1]{\textcolor[rgb]{0.161,0.373,0.58}{\textbf{#1}}}%
\newcommand{\hlkwb}[1]{\textcolor[rgb]{0.69,0.353,0.396}{#1}}%
\newcommand{\hlkwc}[1]{\textcolor[rgb]{0.333,0.667,0.333}{#1}}%
\newcommand{\hlkwd}[1]{\textcolor[rgb]{0.737,0.353,0.396}{\textbf{#1}}}%
\let\hlipl\hlkwb

\usepackage{framed}
\makeatletter
\newenvironment{kframe}{%
 \def\at@end@of@kframe{}%
 \ifinner\ifhmode%
  \def\at@end@of@kframe{\end{minipage}}%
  \begin{minipage}{\columnwidth}%
 \fi\fi%
 \def\FrameCommand##1{\hskip\@totalleftmargin \hskip-\fboxsep
 \colorbox{shadecolor}{##1}\hskip-\fboxsep
     % There is no \\@totalrightmargin, so:
     \hskip-\linewidth \hskip-\@totalleftmargin \hskip\columnwidth}%
 \MakeFramed {\advance\hsize-\width
   \@totalleftmargin\z@ \linewidth\hsize
   \@setminipage}}%
 {\par\unskip\endMakeFramed%
 \at@end@of@kframe}
\makeatother

\definecolor{shadecolor}{rgb}{.97, .97, .97}
\definecolor{messagecolor}{rgb}{0, 0, 0}
\definecolor{warningcolor}{rgb}{1, 0, 1}
\definecolor{errorcolor}{rgb}{1, 0, 0}
\newenvironment{knitrout}{}{} % an empty environment to be redefined in TeX

\usepackage{alltt}
\usepackage{natbib}      % cross referencing bibliography entries
\usepackage{amsmath, amsthm, amsfonts}
\usepackage{graphicx}    % importing graphics into figures
\usepackage{multirow}    % tables with multiple rows
\usepackage{slashbox}    % tables with slash
\usepackage{rotating}    % for vertical words in table
\usepackage{color}       % for textcolor.. (temporary use before submission)
\usepackage{bm}
\usepackage[section]{placeins} % ensure floats do not go into the next section.
\IfFileExists{upquote.sty}{\usepackage{upquote}}{}
\begin{document}

\section{Introduction}
\cite{Albert1992}: Estimates home run rates for 12 players.  Modeled the aging curves as a quadratic function of the number of seasons that a player had played.  


\cite{James1982}: Baseball abstract.  We should get a copy of this.  

\cite{BerryEtAl1999}: Bridging Different Eras in Sports.  Use a nonparametric method for estimating aging curves.  

\cite{Albert1999}:  This is a comment about \cite{BerryEtAl1999}, where Albert again mentions the quadratic aging curve. I think albert like the inerpretating of a quadratic better than the nonparametric curve.   

\cite{Fair2007}: Looks at effects of aging in swimming, running, and chess. 
Mentions Moore1975 (lett's try to get thius.  )

\cite{Fair2008}: JQAS article about aging in baseball.  

\section{Methods}
- Describe what we are doing.  
- Trying to estimate an aging curve using imputation methods to impute the missing years of players' careers due to retirement, drop out, etc.  

\subsection{discrete}
Ideas: 
- Raw aging curve (OPS as measure of performance) imputed with 2L.norm (and covariates)
- Delta method of the aging curve (OPS as measure of performance) imputed with 2L.norm (and covariates)
- Aging curves for different aspects of performance (i.e. Power hitting, OBP, stolen bases?)

- Think about HOW we are actually doing these imputations.  We might need to write our own code to do something more complex than MICE???
- MICE MD?  mice.impute.2l.2stage.pmm?  mice.impute.2l.2stage.norm?

\subsection{continuous}
- A different idea is to take the discrete performance by age and try to fit a continuous curve (Fourier approx or loess or splines or etc.) to that data and then impute that function.  
- Do the ``delta method" to the continuous fitted curve.  



- clustering based on career types.  


\section{Results}

\subsection{Simulation results}
simulate a "real" curve.  We simulate dropout and then check how different types of dropout 

\subsection{Real data example}
Baseball: Different positions? 

Other sports?  
Tennis 
Chess 
Golf
Running
Swimming
Softball

CLustering based on career types?


\section{Conclusions}

The old way of doing aging curves did not account for dropout.  We believe our estimate is better.  And here are the ways that it is different.  
\cite{HeEtAl2011}

\bibliography{aging}
\bibliographystyle{chicago}

\end{document}
